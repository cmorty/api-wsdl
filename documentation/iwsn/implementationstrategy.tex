%------------------------------------------------------------------------------------------------
%------------------------------------------------------------------------------------------------
	\section{Implementation Strategy}
	\label{sec:implstrategy}
%------------------------------------------------------------------------------------------------
%------------------------------------------------------------------------------------------------
This section proposes a strategy for migrating to this version of the WISEBED APIs, assigning clear implementation phases and timeframes for those phases.

We define three major phases of implementation:

\begin{enumerate}

	\item Implementing the WSN API 1.0 as defined in Deliverable 2.2 [{\bf October 2009}]
	\label{step:basic}
	\subitem {Note this can be achieved {\em either} be implementing a `plain' version of 1.0, or by implementing the `fundamentals' of the 2.1 proposal as discussed below {\bf with an adaptor layer to make it 1.0 compliant}}
	
	\item Implementing virtual testbed support using the new create/destroy/enable/disable functions [{\bf 2010}]
	\label{step:vlink}
	
	\item Implementing WiseML support and related functions, including file parsing, and implementing setStartTime() [{\bf 2010}]
	\label{step:definenet}
	
\end{enumerate}

The work required for phases \ref{step:vlink} and \ref{step:definenet} is essentially believed to be the development of a single generic software package which will be able to operate unmodified at any WISEBED site due to the per-site implementations provided in phase \ref{step:basic}.

In more detail, these stages involve the following parts of the WISEBED API proposed in this document:

%------------------------------------------------------------------------------------------------
		\subsection{Fundamentals [October 2009]}
%------------------------------------------------------------------------------------------------
This phase requires implementation of the following functions:

\subsubsection{SessionManager API}
\begin{list}{}{}
\item getInstance() ({\bf NOTE: Will ignore {\em reservation\_id} and return the only instance})
\item free()
\end{list}

\subsubsection{WSN API}
\begin{list}{}{}
\item areNodesAlive()
\item resetNodes()
\item send()
\item flashPrograms()
\item getNetwork() ({\bf NOTE: Using a primitive ML format as in WSN API 1.0, not full WiseML})
\item getVersion()
\end{list}

\subsubsection{Controller API}
NOTE: These functions are obviously for implementation by a controller, which is not required by October 2009; however the {\em calling} of these functions needs to be implemented in the WSN API implementation.
\begin{list}{}{}
\item receiveStatus()
\item receive()
\end{list}

%------------------------------------------------------------------------------------------------
		\subsection{Virtual testbed support [2010]}
%------------------------------------------------------------------------------------------------
The basic underyling technology for this is functional, but needs to be tied into the API calls and to Wiselib \& the OSA software.

\subsubsection{WSN API}
\begin{list}{}{}
\item setVirtualLink()
\item destroyVirtualLink()
\item getFilters()
\item enablePhysicalLink()
\item disablePhysicalLink()
\item enableNode()
\item disableNode()
\end{list}

Note that the specification of the behaviour of enableNode / disableNode is not yet fully determined and will be agreed at a later date.

%------------------------------------------------------------------------------------------------
		\subsection{WiseML-based network definition and monitoring [2010]}
%------------------------------------------------------------------------------------------------
This includes parsing a WiseML file and invoking the other necessary functions of the WISEBED API as a result.

\subsubsection{WSN API}
\begin{list}{}{}
\item getNetwork() ({\bf NOTE: Returning a WiseML-conforming document})
\item defineNetwork()
\item describeCapability()
\item getNeighbourhood()
\item getPropertyValueOf()
\item setStartTime()
\end{list}